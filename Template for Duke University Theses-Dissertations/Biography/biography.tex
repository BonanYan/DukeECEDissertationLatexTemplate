\biography

Bonan Yan is a PhD candidate at the Department of Electrical and Computer Engineering, Duke University. 
He received B.S. degree in Electronics and Information Engineering from Beihang University, China in 2014 and M.S. degree in Electrical and Computer Engineering from University of Pittsburgh, USA in 2017. 
His research interests are application-specific computing schemes with emerging nonvolatile memories. 
His expertise is on the VLSI design of controllers and interface circuits for resistive memory devices, e.g. MRAM and RRAM.
He has developed a series of monolithically integrated RRAM-CMOS mixed-signal in-memory computing engines for machine learning hardware acceleration since 2015.  

He has authored or co-authored 25 papers in top-tier conferences and journals,
including VLSI~\cite{yan2019rram}, IEDM~\cite{yan2017understanding,yan2019designing}, DAC~\cite{liu2015spiking,yan2018neuromorphic}, ICCAD~\cite{yan2017closed}, DATE~\cite{yan2018exploring}, ASP-DAC~\cite{yin2017low}, ISCAS~\cite{yan2016neuromorphic}, ISVLSI~\cite{liu2016memristor}, Advanced Intelligent Systems~\cite{yan2019resistive}, SPIE \cite{taylor2020highly}, \textit{etc}. 
He is serving as a reviewer of IEEE journals, including TCAD, T-ED, TVLSI, Embedded Systems Letters, ACM journal JETC, Elsivier journals, including Neurocomputing and Integration and conferences including DAC, ISCAS and AICAS.  He also designed the symposium logo for ISVLSI, 2016 and served as the web chair of ICCAD HALO workshop, 2019.



